\section{Conclusion and Future Work}
We have introduced offer and resource revocation in order to improve fairness when using Apache Mesos
to manage clusters. Deploying offer revocation to timeout excessively indecisive frameworks permits the
system to recover otherwise wasted resources and provide more offers to more frameworks. We implemented
resource revocation to allow frameworks to obtain more reliable guarantees of system resources. This
extension allows frameworks to put hard limits on task latencies and thus be able to fulfill SLAs. We
found that since offer based DRF in Mesos already did a reasonable job in providing fairness of
resources, it was sufficient to focus revocation on fulfilling needs and not desires of frameworks. This
ultimately minimized goodput loss due to revocation.

We evaluated our revocation mechanisms on a small cluster using synthetic frameworks designed to model
problem scenarios that arise in a production cluster. Our results show that our revocation strategy
works well to ensure fairness by minimizing task latencies at the cost of minimal goodput loss. Resource
revocation was also able to maintain better resource stability and consistencies to participating
frameworks. Greedy frameworks were prevented from locking out undersubscribed frameworks.
Undersubscribed frameworks are able to get a resonable slice of the system without fear of being edged
out the system by greedy frameworks.
