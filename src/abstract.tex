\section{Abstract}
\textit{We demonstrate how adding resource revocation to Mesos allows the
system to provide latency and resource guarantees to frameworks. Mesos, which
uses dominant resource fairness to offers of new resources, was designed
initially for primarly MapReduce-like and found to provide weak guarantees with
more general workloads. This project resolved this issue by allowing frameworks
to explicitely state resource minimums needed to constrain their task latency
and then use these constraints to guide Mesos in revoking resources.  This
solution both minimizes work lost from revocation and leverages the
pre-existing DRF algorithm in Mesos to already provide long-term fairness.
Offer revocation after a timeout was employed to minimize resource waste due to
indecisive frameworks. These adjusments were evaluated on a single machine
using synthetic benchmarks designed to simulate problem scenarios that occur
when running Mesos in production on large 1000+ node clusters. Evalutation
demonstrated frameworks running latency sensitive tasks were able to start new
jobs on a fully-utilized Mesos cluster employing revocation as if they were the
only framework in the system even in the presence of resource-heavy or
indecisive frameworks.}
