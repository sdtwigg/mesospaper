\section{Abstract}
\textit{
Apache Mesos is a cluster manager that uses the dominant resource fairness algorithm in an
attemt to fairly distribute resources amongst participating frameworks. Unfortunately, while
the resource offer based model works well for the MapReduce-like tasks the system was initially
designed for, Mesos cannot reliable provide good latency and fairness guarantees when run on
systems with more general workloads. This paper explored the use of resource revocation wherein
Mesos intentionally terminates tasks of oversubscribed frameworks to free up resources in order
to provide these more reliable latency guarantees. A simple extension of the communication
model allowing frameworks to state minimum resource needs in order to guide Mesos in making
reasonably conservative revocation decisions. This granted Mesos the ability to make these
stronger guarantees while minimizing the amount of work lost and corresponding resource waste
due to revocation. Adjustments to the offer model were also made to prevent frameworks from
typing up resources by failing to respond to offers in a timely manner by forcing offers to
timeout. Changes were evaluated on a small 8 core cluster using synthetic benchmarks
designed to simulate problem scenarios that occur when running Mesos in production on large
1000+ node clusters. Evalutation demonstrated frameworks running latency sensitive tasks were able to
start new jobs on a fully-utilized Mesos cluster emplyoing revocation as if they were the only
framework in the system. The offer revocation changes allowed Mesos to make more offers to more
frameworks even in the presence of indecisive clients. These offer model changes were, as
desired, found to not have an effect on systems with well-behaved decisive frameworks.
}
