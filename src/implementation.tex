\section{Implementation}
In this paper, we contribute two types of revocation mechanisms to mesos: offer revocation and resource
revocation. The idea behind offer revocation is that Mesos will give frameworks adequate time to respond
to an offer it has made. If a framework fails to respond in time, Mesos will forcibly take back the
resources it had to tie down to make the offer. With resource revocation, Mesos periodically searches
for frameworks that have gone over their guaranteed share. Then, if there are frameworks under their
guaranteed share, Mesos will forcibly take the over allocated resources from the frameworks that are
over their guaranteed share and give the resources to the frameworks under their guaranteed share. We
go into a detailed discussion of our modifications to Mesos below.

\subsection{Offer Revocation}
Since offer revocation does not involve any acknowledgements from a framework, we did not have to change
the communication interface between Mesos and a framework. On the Mesos Master side, an offer revocation
looks very much like a situation in which a framework has rejected offer so we were able to resuse a
lot of the mechanisms that were already in place on Mesos Master. The only major addition we made to
Mesos Master is to have Mesos Master start a separate timer task whenever it makes an offer. When the
timer goes off, it will place a offer revoked message onto Mesos Master's event queue. When the event
handler processes the message, it will put the resources in the offer back on the list of allocatable
offers. Note that there is zero possiblity for a race condition in which the offer revocation timer goes
off at the same time that a framework responded to an offer. One of those messages, either the offer
revocation message or the offer response message, will be enqueued first. If the event handler sees 
the offer response message first, then the offer revocation message is ignored. If it sees the offer
revocation message first, then the offer response message is ignored and the framework is notified. On
the framework side, the framework will see that its tasks were never launched onto a Mesos Slave. It
is up to the framework to do the appropriate thing (e.g. add the task back to its task queue).

We did not settle on a offer timeout time since we were not sure how long it would take a well behaved
framework to respond to an offer. Instead, we start the timeout time at some small time. If a framework
fails to respond to resource offer before the timer goes off, we gradually increase the timeout time
up to a maximum. 

\subsection{Resource Revocation}
Resource revocation is much more complicted. At every heartbeat, stock Mesos will perform DRF to
determine which framework to make a resource offer to. We modified Mesos to also check the resource
utlizations of all the frameworks at every heartbeat. Mesos will calculate the amount of resources that
are allocatable ($R_a$), the amount of resources by which each framework is over its guaranteed
share ($R_o$), and the amount of resources by which each framework is under its guaranteed share 
($R_u$). Then if 
\begin{align*}
  R_u > R_a
\end{align*}
Mesos will begin to revoke from the all the frameworks that have gone over their guaranteed share until
the total amount of resources revoked ($R_r$) is such that
\begin{align*}
  R_u = R_a + R_r.
\end{align*}
Keep in mind three things. First, Mesos will only revoke resources from frameworks to bring them down to
their guaranteed share and never under their guaranteed share. Second, Mesos will allow a framework to 
be over its guaranteed share as long as all the other frameworks in their system have met their
guaranteed share. Finally, we assume that it is possible to revoke resources to bring everyone to at
least meet their guarantee. If this is impossible to do, then that means the system is oversubscribed,
a problem that resource revocation is not meant to solve.

Once Mesos has determined the amount of resources to revoke from each framework, it will then determine
which slave these frameworks are currently running tasks on. It will then figure out how many tasks to
kill on each slave such that enough resources are freed. To make this a fast process, we simply have
Mesos kill the most recently launched tasks. Once it has figured out which tasks to kill, Mesos sends
kill task messages to the slaves. Upon receiving these messages, the slaves will immediately kill the
corresponding task and notify Mesos Master and corresponding framework that a task has been killed. On
receipt of this message, Mesos Master will add the resources that the task was using back to the list
of allocatable resources. 
