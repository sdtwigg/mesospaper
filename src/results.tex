\section{Results}
\label{sec:results}
\begin{figure*}[hb]
  \centering

  \begin{subfigure}[t]{0.475\tw}\centering
    \includegraphics[width=\textwidth]{../results/control_greedy_goodput.png}
    \caption{Goodput Results for Framework 1}
    \label{fig:greedy-goodput}
  \end{subfigure}%
  \hfill
  \begin{subfigure}[t]{0.475\tw}\centering
    \includegraphics[width=\textwidth]{../results/control_model_goodput.png}
    \caption{Latency Results for Framework 2}
    \label{fig:model-goodput}
  \end{subfigure}%

  \caption{\textbf{Goodput Results For Frameworks 1 and 2}
  Figure~\ref{fig:greedy-goodput} and Figure~\ref{fig:model-goodput} plots the goodput results for 
  Framework 1 and Framework 2 respectively. The black lines correspond to the case where the frameworks
  are running in isolation. The blue lines correspond to Scenario A while the red lines correspond to
  Scenario B. Solid lines correspond to the case where resource revocation is used while $+$ lines
  correspond to the case when no resource revocation is used.
  }
  \label{fig:control-goodput}
\end{figure*}

\begin{figure*}[hb]
  \centering

  \begin{subfigure}[t]{0.475\tw}\centering
    \includegraphics[width=\textwidth]{../results/150_goodput.png}
    \caption{Goodput Results for Scenario A}
    \label{fig:150-goodput}
  \end{subfigure}%
  \hfill
  \begin{subfigure}[t]{0.475\tw}\centering
    \includegraphics[width=\textwidth]{../results/150_fw_latency.png}
    \caption{Latency Results for Scenario A}
    \label{fig:150-latency}
  \end{subfigure}%

  \caption{\textbf{Goodput and Latency Results for Scenario A}
  In Figure~\ref{fig:150-goodput}, black lines correspond to the system's goodput, blue lines
  correspond to Framework 1's goodput, and red lines correspond to Framework 2's goodput. For the system
  goodput result, the $+$ lines correspond to the case without resource revocation and the dashed line
  corresponds to the cae with resource revocation. For the frameworks, the $+$ lines correspond to the
  case without resource revocation while the solid lines correspond to the case with resource
  revocation.
  Figure~\ref{fig:150-latency} plots Framework B's task latency when it is running in isolation
  (the black line), when it is running without resource revocation (the blue line), and when it is
  running with resource revocation (the red line).
  }
  \label{fig:150-goodput-latency}
\end{figure*}

\begin{figure*}[hb]
  \centering

  \begin{subfigure}[t]{0.475\tw}\centering
    \includegraphics[width=\textwidth]{../results/30_goodput.png}
    \caption{Goodput Results for Scenario B}
    \label{fig:30-goodput}
  \end{subfigure}%
  \hfill
  \begin{subfigure}[t]{0.475\tw}\centering
    \includegraphics[width=\textwidth]{../results/30_fw_latency.png}
    \caption{Latency Results for Scenario B}
    \label{fig:30-latency}
  \end{subfigure}%

  \caption{\textbf{Goodput and Latency Results for Scenario B}
  Figure~\ref{fig:30-goodput} shows that there is still a fair amount of goodput loss in this scenario
  but the amount loss is much less than the amount in Scenario A. This is due to because the tasks
  that are killed have only made 30 seconds of progress whereas in Scenarios A the killed tasks have
  made 150 seconds of progress. Note that in this scenario, the second framework will experience
  a latency 2 orders of magnitude larger due to the fact, without resource revocation, they would have
  to wait even longer before they are given a slot in the system.
  }
  \label{fig:30-goodput-latency}
\end{figure*}

\subsection{Experimental Setup}
Unfortunately, we did not have access to a large cluster to run Mesos on. We instead ran our
experimentation on 8-core 2.67 GHz Intel Xeon processor with 49 GB of usable memory. In all of our
experiments, we used 1 instance of a Mesos Master managing 1 Slave in charge of a 8 slot ``cluster''.
Due to compute resource limitations, the experimental setup is not meant to be representative of a real
Mesos cluster. However, since we designed our synthetic frameworks to highlight specific problem 
scenarios in a Mesos cluster and not for big-picture catch-all experiments, the experimental results
are still highly beneficial.

In our experiments, we define goodput to be the amount of time that a framework was able to use a
resource slot to make progress that is eventually saved. For example, if a task occupies a slot for
30 seconds and after 30 seconds it has completed, then that task has made 30 seconds of goodput over
the past 30 seconds. If a task occupies a slot for 30 second and after 30 seconds it was prematurely
terminated (because it was killed, for example), then that task has made 0 second of goodput over the
past 30 seconds. We measure latency to the difference in time between when a task enters a task queue
and when the task completes. If, for example, a task enters framework's task queue at time 100 seconds,
is launched on a slaved at time 102 seconds, and then completes at time 130 seconds, it latency will be
30 seconds (130 seconds - 100 seconds). 

\subsection{Offer Revocation}
The goal behind offer revocation is to minimize the impact that malicious or poorly coded frameworks
have on the system. In our experiments, if all the frameworks in the cluster are well behaved, then
offer revocation has no effect on the cluster at all, which is the desired behavior. We also introduced
frameworks that take a lengthy amount of time to make a decision on an offer as well as frameworks that
never respond to offers. With offer revocation, well behaved frameworks were able to make forward
progress.

\subsection{Resource Revocation}
For this experiment, we designed two frameworks. The first framework, Framework 1, models a greedy
data analysis framework that wants as much compute resources as possible. It launches long lived tasks
that lasts 3 minutes. If allowed to run in isolation, it will easily take up a large share of the 
cluster. In our experimental setup, we allowed Framework 1 to take up all 8 slots. Framework 2, on the
other hand, models reactive realtime jobs. Every 15 seconds it generates and launches 2 short-lived 
tasks spanning 30 seconds each. If allowed to run in isolation, this framework will only use up two 
slots in the cluster. 

Before diving into our experiments, we first allow Frameworks A and B to run in the cluster isolation to
verify that they behave as intended and to have a good reference point for our experiments. The solid
black lines in Figure~\ref{fig:greedy-goodput} and Figure~\ref{fig:model-goodput} correspond to 
achievable goodput results when the frameworks are allowed to run by themselves. As expected, 
Framework 1 will eat up all 8 slots while Framework 2 will never take up more than 2 slots even if 
Mesos offers it more resources. The solid black lines in Figure~\ref{fig:150-latency} and
Figure~\ref{fig:30-latency} corresponds to the achievable latency when Framework 2 is allowed to run by
itself in the cluster.

We closely examined two extreme corners of operation. Figure~\ref{fig:150-goodput-latency} documents our
results for Scenario A. In this scenario, we have Framework 1 enter the system first. Since it is the 
only framework in the system, it quickly takes up all 8 slots in the system. We then have Framework 2
enter the system 150 seconds later. We first examine how our system behaves without resource revocation.
We'll then show how resource revocation improves the performance of our system. Recall that Framework's
A task each last 3 minutes. This means that without resource revocation, Framework 2 would only have to
wait 30 seconds for Framework 1's task to finish at which point Mesos can then make an offer to 
Framework 2. Furthermore, since Framework 2 generates tasks every 15 seconds, regardless of whether or
not Mesos is able to make it a resource offer, by the time Mesos is able to make it an offer, 
Framework 2 will have a pile up of tasks in its task queue. It will thus grab a larger share of the 
system. This behavior can be seen by the red $+$ line in Figure~\ref{fig:150-goodput}. Notice that at 
180s, this line will jump up to 3. It will then quickly drop to 1. This is because Framework 2 only 
requires 2 slots to keep its task queue empty and having more slots means that it will complete tasks 
faster than it can generate tasks. It will therefore relinquish two of its three slots to Framework 1.
This will cause the tasks in its task queue to pile up again. So then at time 360, when Framework 1's
second set of tasks completes, Framework 2 will take up four slots in the system. This behavior can
also be seen by Framework 2's latency (the solid blue line in Figure~\ref{fig:150-latency}). Notice how
initially Framework 2's tasks have latency an order of magnitude larger than the control case (the
solid black line). However, this latency will quickly drop (because Framework 2 has a much larger share
of the system than it needs) before spiking up again (because it has a smaller share of the system
than it needs).

Now we'll examine what happens when we allow Mesos to revoke resources. The solid blue and red lines
indicate the behaviors of Framework 1 and Framework 2 with revocation respectively. First, notice that
Framework 2 immediately starts at time 150, when it enters the system whereas before it was not able
to start until time 180, as indicated by the red $+$ line. Furthermore, Framework 2's goodput remains
steady throughout its run, fully utilizing its two slots in the system. And finally, Framework 2's tasks
experiences latency that matches closely with the control case (compare the red line to the black line
in Figure~\ref{fig:150-latency}). Although resource revocation improves Framework 2's performance, it 
also introduces some drawbacks into the system that were not previously there. Compare the system's 
goodput with revocation (the dashed black line in Figure~\ref{fig:150-goodput}) to the system's goodput
without resource revocation (the $+$ black line in Figure~\ref{fig:150-goodput}). Notice that there is a
drop in goodput from time 0 to time 150. This is due to the fact that two of Framework 1's tasks are 
killed at time 150. Thus any work that those two tasks made from time 0 to time 150 is lost. However,
this is the only time that the system loses goodput.

In our second scenario, Scenario B, we again had Framework 1 come into the cluster first and allow it
to take up the entire cluster. The difference this time is that we had Framework 2 come into the system
much earlier, at 30 seconds. Our results for this experiment are in Figure~\ref{fig:30-goodput-latency}.
Again, we first examine the performance of our system without resource revocation. Since Framework 2
enters the system at 30 seconds and since Framework 1's tasks each take 3 minutes to complete, Framework
B has to wait 150 seconds before it is receives an offer from Mesos. At this time, Framework 2 will have
built up a lot of tasks in its task queue. So when Framework 2 finally gets an offer from Mesos at time
180 seconds, it will take up four slots, demonstrated by the red $+$ line in 
Figure~\ref{fig:30-goodput}. Similar to the behavior observed in Scenario A, Framework 2 will again
only hold on to this large share of the cluster for a small period of time. Note; however, that in
Scenario B, Framework 2 will hold onto this larger share of the cluster for a much longer period of 
time due to the fact that more tasks will have accumulated in its task queue. Finally, in both
scenarios, Framework 2's tasks will experience similar latency behaviors. The major difference is that
in Scenario B, Framework 2's tasks will have latency that are 2 orders of magnitude larger than the
control (compare solid black lines in Figure~\ref{fig:150-latency} and Figure~\ref{fig:30-latency}).

With resource revocation, Framework 2 will have much better performance in terms of goodput and latency.
Examing the solid red line in Figure~\ref{fig:30-latency}, we can see that Framework 2 starts making
goodput right as it enters the cluster instead of having to wait until time 180 seconds when Framework
A's tasks will complete. Figure~\ref{fig:30-latency} also shows that Framework 2's task's latency
matches closely with that of the control (compare the black line to the red line). Again, resource
revocation does come with a drawback. Comparing the dashed black line to the $+$ black line in
Figure~\ref{fig:30-goodput}, we can see that system does lose goodput. However, the amount loss is very
neglible compared to the amount lost in Scenario A. This is due to the fact that the tasks that Mesos
killed have only made 30 seconds of progress so only 30 seconds of work was lost.

The idea behind these two experiments was to have Scenario A show the case when resource revocation
could possibly hurt the system more than help while Scenario B would highlight the case when the system
would desperately need resource revocation to ensure fairness. However, our results show that resource
revocation is beneficial in both cases. Specifically, resource revocation keeps the system in a much
more stable state, as demonstrated by Figure~\ref{fig:control-goodput}. Notice that the solid lines in
both Figure~\ref{fig:greedy-goodput} and Figure~\ref{fig:model-goodput} are stable while the $+$ lines
in both figures tend to oscillate. Futhermore, resource revocation keeps the latency of later arriving 
frameworks identical to the latency they would experience if they were running by themselves in the 
cluster. These performance gains outweigh the loss caused by the transient goodput drop introduced when
the newer frameworks enter the cluster.
